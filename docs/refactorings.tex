\documentclass{llncs}

\usepackage{url}
\usepackage{graphicx}
\usepackage{listings}
\usepackage{verbatim}
\usepackage[lined,linesnumbered,algochapter]{algorithm2e}
\usepackage{tikz}
\usetikzlibrary{arrows,automata}
\usepackage{xspace}
\usepackage{todonotes}          % Package for working draft comments

\usepackage{bibgerm,cite}       % Deutsche Bezeichnungen, Automatisches Zusammenfassen von Literaturstellen
\usepackage[ngerman]{varioref}  % Querverweise

\setcounter{secnumdepth}{2}
\setcounter{tocdepth}{3}

% define custom macros for specific formats or names
\newcommand{\uml}[1]{\texttt{#1}}
\newcommand{\cd}{\textsf{Class Diagram}}

\begin{document}
\pagestyle{plain}
\pagenumbering{roman}

\title{fUML Refactoring with EMF\footnote{This work has been created in the context of the course ``Advanced Model Engineering'' (188.952) in SS14.}}

\author{Sebastian Geiger (1127054) \and Kristof Meixner (9725208)}
%\institute{Business Informatics Group\\Vienna Technical University}

\maketitle

\begin{abstract}
In this work we will present some ideas and concepts for refactoring fUML with EMF. The main contribution of this work is the extension of
existing UML refactorings to cover not only the static aspect of UML such as class diagrams but also include refactorings for dynamic
parts such as activity diagrams. In this work we will present basic concepts for refactoring with EMF and show how model semantics can be
preserved through the use of OCL constraints. Finally we conclude with a discussion of EMF.Refactor, which shows how such refactorings
can be included into Eclipse GUIs such as EMF tree editor or Papyrus.
\end{abstract}

\tableofcontents
%\thispagestyle{plain}
\newpage

\pagenumbering{arabic}

\section{Introduction}
fUML adds semantics to UML models that make it possible to create semantically closed models which can be executed on the model level. With
fUML classic refactorings are not enough to refactor those models as they do not support the refactoring of the dynamic aspects of models
such as activity diagrams.

\section{Motivation}


\section{Refactorings Examples}
In this section we will present some general refactorings such as the ``extract superclass'' refactoring.

\section{Refactoring of fUML diagrams}
In this section we will present some general refactorings

\section{Tool chain and implementation}

\subsection{Model refactoring}
Describe our tool chain, how we created models, how we load them, what information of the abstract syntax we use for refactoring, etc.

\subsection{GUI Integration}
Describe what we did with EMF.Refactor.

\section{Related Works}
We have compared our works with several other available papers. In [..] there is a discussion of uml refactings which covers ....

some related works such as ...

\section{Conclusion}
We conclude this paper with...

\section{Bibliographic Issues}

\subsection{Literature Search}

Information on online libraries and literature search, e.g., interesting magazines, journals, conferences, and organizations may be found at \url{http://www.big.tuwien.ac.at/teaching/info.html}.

\subsection{BibTeX}

BibTeX should be used for referencing.

The LaTeX source document of this pdf document provides you with different samples for references to journals~\cite{jour:B2BServices}, conference papers~\cite{proc:TheWebMLApproach}, books~\cite{book:umlatwork}, book chapters~\cite{incoll:ErhardKonrad1992}, electronic standards~\cite{man:BPEL}, dissertations~\cite{phdthesis:manuelWimmer}, masters' theses~\cite{mast:AUMLProfile}, and web sites~\cite{misc:BIGWebsite}. The respective BibTeX entries may be found in the file \texttt{references.bib}. For administration of the BibTeX references we recommend \url{http://www.citeulike.org} or JabRef for offline administration, respectively.

\bibliographystyle{acm}
\bibliography{references}

\end{document}
