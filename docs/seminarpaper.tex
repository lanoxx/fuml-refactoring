\documentclass{llncs}

\usepackage{url}
\usepackage{graphicx}
\usepackage{listings}
\usepackage{verbatim}
\usepackage[lined,linesnumbered,algochapter]{algorithm2e}
\usepackage{tikz}
\usetikzlibrary{arrows,automata}
\usepackage{xspace}
\usepackage{todonotes}          % Package for working draft comments
\usepackage{hyperref}           % Package for hyperlink references in the document

\usepackage{ngerman}
\usepackage[ngerman, english]{babel}
\usepackage{bibgerm,cite}       % Deutsche Bezeichnungen, Automatisches Zusammenfassen von Literaturstellen
\usepackage[ngerman]{varioref}  % Querverweise

\setcounter{secnumdepth}{2}
\setcounter{tocdepth}{3}

% define custom macros for specific formats or names
\newcommand{\uml}[1]{\texttt{#1}}
\newcommand{\cd}{\textsf{Class Diagram}}

\begin{document}
\pagestyle{plain}
\pagenumbering{roman}

\title{Refactoring UML models}


%&&&&&&&&&&&&&&&&&&&&&&&&&&&&&&&&&&&&&&&&&&&&&&&&&&&&&&&&&&&&&&&&&&&&&&&&
% Name and address of the author
%&&&&&&&&&&&&&&&&&&&&&&&&&&&&&&&&&&&&&&&&&&&&&&&&&&&&&&&&&&&&&&&&&&&&&&&&
\author{Kristof Meixner}

\institute{\email{kristof.meixner@fatlenny.net} \\ Registration No. 9725208}
% \institute{Musterweg 12/3/7, 1040 Wien \\ \email{mustermann@tuwien.ac.at} \\ Registration No.: 0748549}

%&&&&&&&&&&&&&&&&&&&&&&&&&&&&&&&&&&&&&&&&&&&&&&&&&&&&&&&&&&&&&&&&&&&&&&&&
% Example for more than one authors
%&&&&&&&&&&&&&&&&&&&&&&&&&&&&&&&&&&&&&&&&&&&&&&&&&&&&&&&&&&&&&&&&&&&&&&&&
%\author{Max Mustermann\inst{1} and Matthias Mustermann\inst{2}}

%\institute{Musterweg 12/3/7, 1040 Wien \\ \email{mustermann@tuwien.ac.at} \\ MatrNr.: 0748549
%\and
%Mustergasse 54/4/3, 1030 Wien \\ \email{matthias@tuwien.ac.at} \\ MatrNr.: 0426553
%}

\maketitle

\begin{abstract}
Over the last twenty years refactoring advanced to a commonly known and used techniques in modern software engineering. We present an overview from the beginning of refactoring in source code to its actual application in model-driven software development. Furthermore we discuss methods that ensure that refactored source code is sill correct.
\end{abstract}

%&&&&&&&&&&&&&&&&&&&&&&&&&&&&&&&&&&&&&&&&&&&&&&&&&&&&&&&&&&&&&&&&&&&&&&&&
% Table of contents
% Activate or deactivate this according to the guideline instructor
%&&&&&&&&&&&&&&&&&&&&&&&&&&&&&&&&&&&&&&&&&&&&&&&&&&&&&&&&&&&&&&&&&&&&&&&&
\tableofcontents
%\thispagestyle{plain}
\newpage

\pagenumbering{arabic}

\section{Refactoring in the beginning}

In his thesis \cite{mast:REFOOF} Opdyke ``defines a set of program restructuring operations'' that ``preserve the behavior of a program'' to increase software quality. This technique became known as refactoring.

The issue he addressed in his work is the problem of changing parts of source code from an object oriented system, grounded in a possibly large code base while also maintaining all the references and dependencies manually. He described this process as ``time consuming, difficult and error prone''. As a solution he proposes ``an approach for providing automated support for the restructuring'', plans to reorganize the source code on an intermediate level without changing the behavior of the program.
 

\newpage
\bibliographystyle{acm}
\bibliography{references}

\end{document}
